\IEEEPARstart{I}{n the} past decades, there have been a large number
of studies of quantum well intermixing (QWI) with the aim to develop
the technique for monolithic integration of optoelectronic devices
and photonic integrated circuits (PICs) applications \cite{JMarsh}.
With this technique, it is possible to selectively modify the
composition variation within a small region in a wafer. The emerged
technology of QWI, such as ion-implantation induced disordering
(IIID) \cite{IIID}, impurity-free vacancy disordering (IFVD)
\cite{IFVD}, sputtered silica-induced intermixing \cite{SiO2},
photoabsorption-induced disordering (PAID) \cite{PAID} and
argon-plasma enhanced quantum well intermixing \cite{ArPlasma}, have
been utilized in wide application for PICs, such as the 40-Gb/s
Widely Tunable Transceivers by UCSB \cite{40G}.

In the meanwhile, theory works of QWI have been carried out which
play an important role in understanding the mechanics and optimizing
the performance of this technology. By far, there has been a widely
accepted theory of QWI, which talks about the optical properties,
including stimulated wavelength, optical absorption, and optical
gain. The simulation results agree quite well with the
experimentally measured values. I will talk about the simulation
process in this paper. The interdiffusion mechanics of QWI, which is
more complicated to understand, will be investigated in the next
term.

This paper is organized as follows. The theoretical analysis of QWI
are investigated in Section II. Fundamental experiments of QWI,
including bandgap energy shift characteristics, waveguiding
characteristics, electrical characteristics and carrier induced
tuning characteristics are described in Section III. Some
applications of QWI, including bandgap tuning of InP-based lasers
and extended cavity laser, are presented and discussed in Section
IV. Section V summarizes the main conclusions of this paper.
