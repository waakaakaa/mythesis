Interdiffusion of atoms across the heterointerface alters the
composition profile across the QW structure. Mathematical models are
needed to describe and calculate the changes in the composition
profile as a function of the duration of interdiffusion. In the
section, we adopt the notation $A_{x}B_{1-x}C_{y}D_{1-y}$ to denote
the chemical formula of a III-V semiconductor material, where A and
B represent the Group III atoms, and C and D represent the Group V
atoms respectively. After interdiffusion the mole fractions of Group
III and V atoms are function of position along the direction of
crystal growth (the $z$-direction) and are denoted by $w(z)$ and
$v(z)$.

The diffusion process is usually described by the Fick's Second Law
in the direction of crystal growth
\begin{equation}
\frac{\partial{C}}{\partial{t}}=D\frac{\partial^2{C}}{\partial{z^2}}
\end{equation}
where $C$ denotes the concentration of the atoms and $D$ is the
diffusion coefficient. The diffusion coefficient of the atoms is
usually assumed to be identical in the quantum well. The
interdiffusion process is characterized by a diffusion length $L_d$,
which is defined as $L_d=\sqrt{Dt}$, where $D$ is the diffusion
coefficient and $t$ is the annealing time of the thermal processing.
Consider a single
$In_{wx}Ga_{1-wx}As_{wy}P_{1-wy}/In_{bx}Ga_{1-bx}As_{by}P_{1-by}$
quantum well, the composition profile of In and As in this QW after
interdiffusion are given by
\begin{eqnarray}
    w(z)&=&wx-\frac{bx-wx}{2} \notag\\
        &&\times \biggr[2-\textrm{erf}\biggr(\frac{L_z+2z}{4L_d^{III}}\biggr)
        -\textrm{erf}\biggr(\frac{L_z-2z}{4L_d^{III}}\biggr)\biggr]
\end{eqnarray}

\begin{eqnarray}
    v(z)&=&wy-\frac{by-wy}{2} \notag\\
        &&\times \biggr[2-\textrm{erf}\biggr(\frac{L_z+2z}{4L_d^{V}}\biggr)
        -\textrm{erf}\biggr(\frac{L_z-2z}{4L_d^{V}}\biggr)\biggr]
\end{eqnarray}

In this case, interdiffusion can occur on both group III and V
sublattices, which are characterized by their diffusion lengths
$L_d^{III}$ and $L_d^V$, respectively. The diffusion length ratio,
i.e., $k=L_d^V/L_d^{III}$, depends on the details of interdiffusion
process and is still an open question which requires further
studies.
